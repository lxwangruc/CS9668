\documentclass[paper=a4, fontsize=11pt]{scrartcl} % A4 paper and 11pt font size

\usepackage[english]{babel} % English language/hyphenation
\usepackage{graphicx,pifont,color} 
\usepackage{amsmath}
\usepackage[byname]{smartref}
%\usepackage{hyperref} %comment out for hardcopy
\usepackage{txfonts}
\usepackage{tocloft}
\usepackage{enumitem}

\usepackage{hyperref}
\usepackage{lhelp}
\usepackage{listings}
\usepackage{float}
\usepackage{xcolor}
\lstset { %
	language=C,
	backgroundcolor=\color{black!5}, % set backgroundcolor
	basicstyle=\footnotesize,% basic font setting
	belowcaptionskip=1\baselineskip,
	breaklines=true,
	frame=L,
	xleftmargin=\parindent,
	showstringspaces=false,
	basicstyle=\footnotesize\ttfamily,
	keywordstyle=\bfseries\color{green!40!black},
	commentstyle=\itshape\color{purple!40!black},
	identifierstyle=\color{blue},
	stringstyle=\color{orange},
	numbers=left,
	stepnumber=1,
}
\usepackage{lipsum} % Used for inserting dummy 'Lorem ipsum' text into the template

\usepackage{sectsty} % Allows customizing section commands
\allsectionsfont{\centering \normalfont\scshape} % Make all sections centered, the default font and small caps

\usepackage{fancyhdr} % Custom headers and footers
\pagestyle{fancyplain} % Makes all pages in the document conform to the custom headers and footers
\fancyhead{} % No page header - if you want one, create it in the same way as the footers below
\fancyfoot[L]{} % Empty left footer
\fancyfoot[C]{} % Empty center footer
\fancyfoot[R]{\thepage} % Page numbering for right footer
\renewcommand{\headrulewidth}{0pt} % Remove header underlines
\renewcommand{\footrulewidth}{0pt} % Remove footer underlines
\setlength{\headheight}{13.6pt} % Customize the height of the header

\numberwithin{equation}{section} % Number equations within sections (i.e. 1.1, 1.2, 2.1, 2.2 instead of 1, 2, 3, 4)
\numberwithin{figure}{section} % Number figures within sections (i.e. 1.1, 1.2, 2.1, 2.2 instead of 1, 2, 3, 4)
\numberwithin{table}{section} % Number tables within sections (i.e. 1.1, 1.2, 2.1, 2.2 instead of 1, 2, 3, 4)

\setlength\parindent{0pt} % Removes all indentation from paragraphs - comment this line for an assignment with lots of text

%----------------------------------------------------------------------------------------
%	TITLE SECTION
%----------------------------------------------------------------------------------------

\newcommand{\horrule}[1]{\rule{\linewidth}{#1}} % Create horizontal rule command with 1 argument of height

\title{	
\normalfont \normalsize 
\textsc{Computer Science Department, UWO} \\ [25pt] % Your university, school and/or department name(s)
\horrule{0.5pt} \\[0.4cm] % Thin top horizontal rule
\huge CS9668\\
Assignment 2 \\ % The assignment title
\horrule{2pt} \\[0.5cm] % Thick bottom horizontal rule
}

\author{Linxiao Wang\\
	250888611} % Your name

\date{\normalsize\today} % Today's date or a custom date

\begin{document}

\maketitle % Print the title

%----------------------------------------------------------------------------------------
%	PROBLEM 1
%----------------------------------------------------------------------------------------

\section*{Problem 1}
\subsection*{Description of the algorithm}
For this question, we can use a similar approach as the second  LeaderElection algorithm introduced in class. In the following description, we refer to the ID in the received message id\_m, and the ID of the current processor id\_p.
\begin{itemize}
	\item[1] Every processor doesn't know whether it is the Leader of the network or not, so it starts to send a message to it's right neighbour with its own ID and a distance d = 1.
	\item[2] After receiving a message from its left neighbour, if id\_m is smaller than id\_p, the processor will discard the message. If id\_m is larger than id\_p, the processor will know that it is not the Leader. If the distance in the message is 1, the processor will send the message back to its left neighbour without a distance, else the processor should forward the message to its right neighbour with the distance minus 1.
	\item[3] After receiving a message from its right neighbour, if id\_m does not equal to id\_p, then the processor should forward the message to its left neighbour, otherwise, the processor should send a message with its own ID with a distance $d = 2d$ to its right neighbour.
	\item[4] If a processor receives a message from its left neighbour and id\_m is equal to id\_p, then the processor knows that it is the Leader. The leader should read the distance in the message(dist), and the number of the processors should be $\#p = d - dist + 1$.
	The leader should send an "END" message with $\#p$ to its right neighbour and return $\#p$.
	\item[5] If a processor receives an "END" message, it knows that it is not the leader and it can read $\#p$ from the message. It should forward the message to its right neighbour and return $\#p$.
\end{itemize}
\subsection*{Implementation}
See file CountProcessorsRing.java.

At the end of the algorithm, every processor will return the number of the processors in the network.
\subsection*{Complexity computation}
We can separate the whole process into phases, in each phase, a message is sent from a processor and come back to the processor from its right neighbour. So starts from phase 0, in phase $i$ , the initial distance in the message is $2^i$ which means there will be $2^{i} * 2 = 2^{i+1}$ rounds in that phase. So the total number of rounds is $\sum_{0}^{k}2^{i+1} = 2^1 + 2^2 + \cdots + 2^{k+1} = 2^{k +2} - 2 = 2^2 * 2^k -2$. When $n >= 2^k$ the program will terminate, to make it simple we assume $n = 2^k$. 

Time complexity is  $2^2 * 2^k -2 + n = 4n - 2 + n = 5n - 2$ which is $O(n)$.



\section*{Problem 2}
\subsection*{Description of the algorithm}
In the following description, we refer to the processor who doesn't have a left neighbour the leftmost node, the processor who doesn't have a right neighbour the rightmost node, all the processors in between the middle nodes. Also, we refer to the ID in the received message id\_m, and the ID of the current processor id\_p.
\begin{itemize}
	\item[1] At the beginning of the algorithm, every processor determines whether it is the rightmost node or leftmost node or a middle node.

	\item[2] Then the leftmost node sends its ID to its right neighbour.

	\item[3] For every middle node, after receiving a message from its left neighbour, the processor send the larger one between id\_m and id\_p to its right neighbour.
	\item[4] When the rightmost node receives a message, it compares the id\_m with id\_p, if id\_m is larger, then the processor knows it is not the leader, else it is the leader. Then the processor send the larger one between id\_m and id\_p to its left neighbour, and return the status.
	\item[5] For every middle node, after receiving a message from its right neighbour, the processor compare id\_m with id\_p, if id\_m is larger, then the processor knows that it is not the leader, else it is the leader. Then the processor forward the message to its left neighbour and return the status.
	\item[6] When the leftmost node receives a message, it compares the id\_m with id\_p, if id\_m is larger, then the processor knows it is not the leader, else it is the leader, then return the status.
\end{itemize}
\subsection*{Implementation}
See the LeaderElectionLine.java file.
\subsection*{Prove of termination}
\begin{itemize}
	\item Because its a line network, so there will be one and only one leftmost processor. Start from the leftmost processor, every processor will send a message to its right neighbour, so the right most processor will receive a message, determine its status and terminate according to step 4 above. 
	\item The rightmost processor will send a message to its left neighbour, and the message will be forwarded to all the left neighbours by all the middle processors. 
	\item According to step 5 and 6, all the middle processors and the leftmost processor will receive a message from their right neighbours, so they all will terminate.
\end{itemize}
\subsection*{Prove of correctness}
\begin{itemize}
	\item Start from the leftmost processor, every processor will send a message contains the largest ID so far. So when the rightmost processor receive a message, it will contain the largest ID from all the processors except the rightmost one. 
	\item The rightmost one can compare the largest one from all the processors from its left with its own, so it will know whether it is the leader or not. 
	\item Then the rightmost processor will send the largest ID among all the processors to its left neighbour, and the largest ID will be forwarded to all the processors. All the other processors will compare that largest ID with their own and know whether they are the leader or not. 
\end{itemize}
\subsection*{Complexity computation}
Here we assume $n$ is the number of the processors in the network.

Start from the leftmost processor, one message will be send per round, until the right most processor get an message. This will take $n - 1$ round and $n - 1$ messages are sent.

Then a message will be sent from the rightmost processor to the leftmost processor, it will take $n - 1$ rounds and $n - 1$ messages are sent.

Time complexity: $2(n - 1) = 2n - 1$ which is $O(n)$.

Communication complexity:  $2(n - 1) = 2n - 1$ which is $O(n)$.

\section*{Problem 3}
\subsection*{Description of the algorithm}
 We refer to the ID in the received message id\_m, the ID of the current processor id\_p and the largest id a processor has seen so far id\_l.
\begin{itemize}
	\item[1] The first step of the algorithm is to find the largest id in each column. So all the processors in the left row will start to send its id to the right. After getting a message from the left, a processor will choose the larger id between id\_m and id\_p and send it to the left. 
	\item[2] When all the processors in the right row gets the largest id from its column, the top-right processor will start to send id\_l to the bottom processor. After getting a message from the top, a processor will send the larger one between id\_m and id\_l to the bottom. 
	\item[3] When the right-bottom processor get a message from the top, it will know the largest id in the whole grid. It can determine it's status and send the message to top and left and return. 
	\item[4] Each processor in the right row will get the largest id and determine its status and send the id to the top and left and return,
	\item[5] Other processors will determine their status when getting the largest id, send it to the left and return.
	\item[6] When the processors in the left row get the largest id, they will determine their status and return.
\end{itemize}
\subsection*{Implementation}
See the LeaderElectionMesh.java file.
\subsection*{Prove of termination}
There are 3 types of processors in this algorithm, the right 
\begin{itemize}
	\item[1] In the first round, the left row processors send messages to the right. In the next $W - 2$ rounds, after receiving a message from the left neighbour, the processors send messages to the right. So after $W - 1$ rounds, the right row processors will receive messages. 
	\item[2] Then the top-right processor starts to send message to the bottom neighbour, and all the processors receiving a message from the top will send one to the bottom neighbour, so then bottom-right processor will receive a message from its top neighbour.
	\item[3] The bottom-right processor send a message to its top neighbour and set the roundLeft to 1. All the other right row processors will receive a message from the bottom and send the message to the top neighbours and set the roundLeft to 1. 
	\item[4] For the right row processors, after sending the message to its top, they prepare a message to sent to its left neighbour.
	\item[5] Processors that receive messages from their right neighbour will forward the message to its left.
	\item[5] All the processors that send a message to the left neighbour at the beginning of a round will set the roundLeft to 0. When roundLeft reaches 0, the processor will stop. Since all the processors other than the left row ones, will receive a message from ????????????????????
\end{itemize}
\subsection*{Prove of correctness}
\begin{itemize}
	\item[1] Start with the left row processors, all the processor will send the largest id so far to its right. So the right row processors will get the largest ids of their columns.
	\item[2] Then start with the top-right processor, each right row processor will send the largest id so far to its bottom neighbour. So that the bottom-right processor will get the largest id of the grid.
	\item[3] A message contains the largest id will be sent from the bottom-right processor to the right row processors, so all the right row processors will see the largest id, after comparing the largest is and their own ids, the right row processors will know their correct status.
	\item[4] Then the right row processors will forward the largest to the left and the messages will be forwarded to all the processors, so all the processors will know whether or not they are the leader.
\end{itemize}
\subsection*{Complexity computation}
\textbf{Time complexity: $O(W + L)$}

Getting the largest id in each column will take W rounds.

Then it will take L rounds for the bottom-right processor to get the largest id in the grid.

It will take L rounds for all the right row processors the get the largest id.

Then it will take W rounds to pass the largest id to the left row.

So the total number of rounds is $2W + 2L$ which is \textbf{$O(W + L)$.}


\textbf{Communication complexity: $O(WL)$}

To get the largest id of each column, it will need $(W-1)\times L$ messages.

Then it will need $L - 1$ messages to get the largest id to the bottom-right.

It will follow the same path backward to send the largest id to all the processors, so there are another $(W-1)\times L + (L - 1)$ messages.

So the total messages that are sent is $2((W-1)\times L + (L - 1)) = 2WL - 2$ which is in the order of \textbf{$O(WL)$.}
\end{document}